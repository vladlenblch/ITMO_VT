\documentclass[a4paper, 10pt]{article}

\usepackage[russian]{babel}
\usepackage[T2A]{fontenc}
\usepackage[utf8]{inputenc}
\usepackage{multicol}
\usepackage{adjustbox}
\usepackage{amsmath}
\usepackage{tikz}
\usepackage{amssymb}
\usepackage{setspace}
\usepackage[dvipsnames]{xcolor}

\usepackage{geometry}
\geometry{top=10mm, bottom=5mm, left=10mm, right=10mm}

\usepackage{graphicx}
\usepackage{wrapfig}

\pagestyle{empty}

\begin{document}

    \newcommand{\bluelineAB}{
        \begin{tikzpicture}
            \draw[line width=1mm, NavyBlue] (0,0) -- (1.5,0);
            \node[above, scale=0.8] at (0,0) {$A$};
            \node[above, scale=0.8] at (1.5,0) {$B$};
        \end{tikzpicture}
    }

    \newcommand{\redlineCA}{
        \begin{tikzpicture}
            \draw[line width=1mm, BrickRed] (0,0) -- (1.5,0);
            \node[above, scale=0.8] at (0,0) {$C$};
            \node[above, scale=0.8] at (1.5,0) {$A$};
        \end{tikzpicture}
    }

    \newcommand{\bluelineDE}{
        \begin{tikzpicture}
            \draw[line width=0.5mm, NavyBlue] (0,0) -- (1.5,0);
            \node[above, scale=0.8] at (0,0) {$D$};
            \node[above, scale=0.8] at (1.5,0) {$E$};
        \end{tikzpicture}
    }

    \newcommand{\redlineFD}{
        \begin{tikzpicture}
            \draw[line width=0.5mm, BrickRed] (0,0) -- (1.5,0);
            \node[above, scale=0.8] at (0,0) {$F$};
            \node[above, scale=0.8] at (1.5,0) {$D$};
        \end{tikzpicture}
    }

    \newcommand{\blacklineBC}{
        \begin{tikzpicture}
            \draw[line width=1mm, black] (0,0) -- (1.5,0);
            \node[above, scale=0.8] at (0,0) {$B$};
            \node[above, scale=0.8] at (1.5,0) {$C$};
        \end{tikzpicture}
    }

    \newcommand{\orangelineEF}{
        \begin{tikzpicture}
            \draw[line width=0.5mm, Dandelion] (0,0) -- (1.5,0);
            \node[above, scale=0.8] at (0,0) {$E$};
            \node[above, scale=0.8] at (1.5,0) {$F$};
        \end{tikzpicture}
    }

    \newcommand{\orangeangle}{
        \begin{tikzpicture}
            \draw[Dandelion,fill=Dandelion] (0, 0) -- (0.25, -0.9) arc(-60:-115:0.7) -- cycle;
            \node[above] at (0,0) {\tiny$A$};
            \node[below] at (0.25, -0.9) {\tiny$B$};
            \node[below] at (-0.45, -0.9) {\tiny$C$};
        \end{tikzpicture}
    }

    \newcommand{\blackangle}{
        \begin{tikzpicture}
            \draw[black,fill=black] (0, 0) -- (0.25, -0.9) arc(-60:-115:0.7) -- cycle;
            \node[above] at (0,0) {\tiny$D$};
            \node[below] at (0.25, -0.9) {\tiny$E$};
            \node[below] at (-0.45, -0.9) {\tiny$F$};
        \end{tikzpicture}
    }
    
    \begin{minipage}{0.59\textwidth}
        \hfill \Large{КНИГА I ПРЕДЛ. XXV. ТЕОРЕМА} \hfill \large{49}
        \vspace{3mm}
        \begin{wrapfigure}[5]{l}{0.19\textwidth}
            \includegraphics[width=0.2\textwidth]{e.jpeg}
        \end{wrapfigure}
        
        \\
        \begin{spacing}{1.2}
            \large сли \textit{у двух треугольников две стороны \bluelineAB и \redlineCA соответственно равны
            двум сторонам \bluelineDE и \redlineFD другого, но основания неравны, то угол над большим
            основанием \blacklineBC одного треугольника меньше угла под меньшим \orangelineEF другого.}
        \end{spacing}
        
        \vspace*{1em}
        
        \begin{center}

            \raisebox{-1.3em}{\orangeangle} \Large{=} , \Large{>} или \Large{<} \raisebox{-1.3em}{\blackangle} \\

            \raisebox{-1.3em}{\orangeangle} не равен \raisebox{-1.3em}{\blackangle}, \\ 

            поскольку если \raisebox{-1.3em}{\orangeangle} \Large{=} \raisebox{-1.3em}{\blackangle}, то \\

            \blacklineBC \Large{=} \orangelineEF (пр. I.\raisebox{-0.2em}{4}), \\

            что противоречит гипотезе; \\
            
            \vspace*{1em}

            \raisebox{-1.3em}{\orangeangle} не меньше \raisebox{-1.3em}{\blackangle}, \\

            поскольку если \raisebox{-1.3em}{\orangeangle} \Large{<} \raisebox{-1.3em}{\blackangle}, \\

            то \blacklineBC \Large{<} \orangelineEF (пр. I.\raisebox{-0.2em}{24}), \\

            что противоречит гипотезе. \\

            $\therefore$ \raisebox{-1.3em}{\orangeangle} \Large{>} \raisebox{-1.3em}{\blackangle}. \\

        \end{center}

        \begin{flushright}

            ч.т.д.

        \end{flushright}
        
    \end{minipage}
    \hfill
    \begin{minipage}{.3\textwidth}

        \begin{center}

            \begin{tikzpicture}
                \coordinate [label=above:{\tiny$A$}] (A) at (-2,3);
                \coordinate [label=below left:{\tiny$C$}] (C) at (-4,-1);
                \coordinate [label=below right:{\tiny$B$}] (B) at (-1,0);

                \draw[Dandelion,fill=Dandelion] (-2, 3) -- (-1.75, 2.1) arc(-60:-115:0.7) -- cycle;
            
                \draw[line width=1mm, BrickRed] (C) -- (A);
                \draw[line width=1mm, NavyBlue] (A) -- (B);
                \draw[line width=1mm, black] (C) -- (B);
                
            \end{tikzpicture}

            \vspace*{2em}
            
            \begin{tikzpicture}
                \coordinate [label=above:{\tiny$D$}] (D) at (-2,3);
                \coordinate [label=below left:{\tiny$F$}] (F) at (-4,-1);
                \coordinate [label=below right:{\tiny$E$}] (E) at (-1,0);

                \draw[black,fill=black] (-2, 3) -- (-1.75, 2.1) arc(-60:-115:0.7) -- cycle;
                
                \draw[line width=0.5mm, BrickRed] (F) -- (D);
                \draw[line width=0.5mm, NavyBlue] (D) -- (E);
                \draw[line width=0.5mm, Dandelion] (F) -- (E);
            
            \end{tikzpicture}

            \vspace*{22em}
            
        \end{center}
        
    \end{minipage}

\end{document}
